


%ICFO poster template
%====================
%
%This is a LaTeX poster template designed to match the existing (powerpoint) poster template for ICFO - The Institute of Photonic Sciences, Barcelona.
%
%This template uses the baposter.cls LaTeX class, available from http://www.brian-amberg.de/uni/poster/, and it requires LuaLaTeX, and the Swiss 721 font, to compile correctly in its standard form. To compile, the Swiss 721 font should be installed in the standard system directory, and (all being well) the fontspec package will get it from there. Given that, compilation is as simple as
%
%    lualatex poster.tex
%
%Given the dependence on fontspec, compiling with pdflatex will not work. If lualatex is not available, remove the fontspec dependence and default to the helvetica-based sans-serif fonts below it; it will compile, though it won't be too pretty.
%
%--------------------
%
%Created by Emilio Pisanty, 2017-09-20, based on existing ICFO materials


\documentclass[a0paper,portrait]{baposter}

\usepackage{relsize}         % For \smaller
\usepackage{url}             % For \url
\usepackage{epstopdf}	     % Included EPS files automatically converted to PDF to include with pdflatex
\usepackage{xcolor}
\usepackage[export]{adjustbox}
\usepackage{setspace}
\usepackage{multicol}
\usepackage{amsmath,amssymb}
\usepackage{mathrsfs}        % For \mathscr



%%% Global Settings %%%%%%%%%%%%%%%%%%%%%%%%%%%%%%%%%%%%%%%%%%%%%%%%%%%%%%%%%%%

\graphicspath{{Images/}}	% Root directory of the pictures 
\tracingstats=2			% Enabled LaTeX logging with conditionals

%%% Color Definitions %%%%%%%%%%%%%%%%%%%%%%%%%%%%%%%%%%%%%%%%%%%%%%%%%%%%%%%%%

\definecolor{bordercol}{RGB}{40,40,40}
\definecolor{headerfontcol}{RGB}{0,0,0}
%
\definecolor{icfogreen}{RGB}{161,207,15}
\definecolor{icfogray1}{RGB}{233,233,233}
\definecolor{icfogray2}{RGB}{192,192,192}
\definecolor{icfogray3}{RGB}{116,116,116}
\definecolor{authorgray}{RGB}{89,89,89}

%%%%%%%%%%%%%%%%%%%%%%%%%%%%%%%%%%%%%


%%% Top margin to zero to make the title box reach the top of the paper
\geometry{tmargin=0mm,bmargin=3.5mm,lmargin=9mm,rmargin=9mm}

%%% Sans-serif maths
\usepackage{arev}

%%% other options that work OK:
%\usepackage[helvet]{sfmath}
%\usepackage[tx]{sfmath}
%\usepackage[px]{sfmath}
%\usepackage[cm]{sfmath}
%%% Also try interchanging the order with the fontspec package, and removing the no-math option from the latter


%%% The font considerations here *require* compilation with lualatex instead of pdflatex.
\usepackage[no-math]{fontspec}
\setmainfont{swiss.ttf}[
   BoldFont = swissb.ttf ,
   ItalicFont = swissi.ttf ,
   BoldItalicFont = swissbi.ttf
 ]
%
\newcommand{\titlefont}{\bf \fontsize{26pt}{28pt} \selectfont}
\newcommand{\authorfont}{\bf \Large}
\newcommand{\headerfont}{\fontspec{swissk.ttf}\fontsize{14pt}{16pt}\selectfont}
\newcommand{\subheaderfont}{\fontspec{swissk.ttf}\fontsize{11pt}{13pt}\selectfont}
\newcommand{\acknowledgementsfont}{\fontsize{8.5pt}{10.5pt}\selectfont}
\newcommand{\footerfont}{\fontsize{5pt}{7pt}\selectfont}

%%%% In the absence of lualatex, use instead e.g.
%%\usepackage[scaled]{helvet}
%%\renewcommand\familydefault{\sfdefault} 
%%\usepackage[T1]{fontenc}
%%
%%\newcommand{\titlefont}{\bf \Huge}
%%\newcommand{\authorfont}{\bf \Large}
%%\newcommand{\headerfont}{\bf \Large}
%%\newcommand{\subheaderfont}{\bf \large}
%%\newcommand{\acknowledgementsfont}{\small}
%%\newcommand{\footerfont}{\footnotesize}





%%%%%%%%%%%%%%%%%%%%%%%%%%%%%%%%%%%%%%%%%%%%%%%%%%%%%%%%%%%%%%%%%%%%%%%%%%%%%%%
%%% Document Start %%%%%%%%%%%%%%%%%%%%%%%%%%%%%%%%%%%%%%%%%%%%%%%%%%%%%%%%%%%%
%%%%%%%%%%%%%%%%%%%%%%%%%%%%%%%%%%%%%%%%%%%%%%%%%%%%%%%%%%%%%%%%%%%%%%%%%%%%%%%

\begin{document}

\typeout{Poster rendering started}

%%% Setting Background Image %%%%%%%%%%%%%%%%%%%%%%%%%%%%%%%%%%%%%%%%%%%%%%%%%%
\background{} %%% No background

%%% General Poster Settings %%%%%%%%%%%%%%%%%%%%%%%%%%%%%%%%%%%%%%%%%%%%%%%%%%%
\begin{poster}{
    grid=false,
    borderColor=bordercol,
    headerColorOne=icfogreen,
    headerColorTwo=icfogreen,
    headerFontColor=headerfontcol,
    boxColorOne=icfogray1,
    headerheight=0mm,
    boxheaderheight=2em,
    headershape=rectangle,
    headerfont=\headerfont,
    textborder=none,
    headerborder=none,
    background=plain,
    bgColorOne=white,
    boxshade=plain,
    columns=2
}
{}{}{}{}  %%%% The standard baposter title settings are not flexible enough, so this sets them empty, and then re-creates them anew inside a standard headerbox.



%%%%%%%%%%%%%%%%%%%%%%%%%%%%%%%%%%%%%%%%%%%%%%%%%%%%%%%%%%%%%%%%%%%%%%%%%%%%%%%
%%%%   Formalities %%%%%%%%%%%%%%%%%%%%%%%%%%%%%%%%%%%%%%%%%%%%%%%%%%%%%%%%%%%%
%%%%%%%%%%%%%%%%%%%%%%%%%%%%%%%%%%%%%%%%%%%%%%%%%%%%%%%%%%%%%%%%%%%%%%%%%%%%%%%

%%%%%%%  Title box.
\headerbox{}{
name=titlebox, span=2, column=0, row=0, boxheaderheight=0mm, 
	headerColorOne=icfogreen,
	headerColorTwo=icfogreen,
    boxColorOne=icfogreen, 
    boxColorTwo=icfogreen,
    boxshade=shadelr
}{
\vspace{5mm}
\begin{minipage}{0.99\textwidth}
\begin{tabular}{p{0.025\textwidth}p{0.16\textwidth}p{0.015\textwidth}p{0.72\textwidth}}
&
\vspace{0.5mm}
\includegraphics[width=0.16\textwidth,valign=T]{Logos/ICFO.png}
&&
\vspace{0pt}
\begin{minipage}{0.72\textwidth}
\raggedright
\setstretch{2.7}
{ \titlefont 
%
Title of the project will go here, font Swiss721 Black BT in size huge and justified to the left
%
}\\[3mm]
\setstretch{1.5}
\textcolor{authorgray}{
\authorfont
%
Names of authors will go here in dark grey using the font Swiss721 Bold BT in size 48 and justified to the left
%
}\\[2mm]
\setstretch{1.25}
\textcolor{authorgray}{
\small
%
Affiliation, departments/groups\\
Affiliation, departments/groups
%
}
\end{minipage}
\end{tabular}
\vspace{5mm}
\end{minipage}
}

%%%%%%%%%%%%%%%%%%%%%%%%%%%%%%%%%%%%%%%%%%%%%%%%%%%%%%%%%%%%%%%%%%%%%%%%%%%%%%%

%%% Bottom logos. Typeset at the start because (i) they're a formality, and (ii) the bottom content boxes require this to be defined for the above=bottomlogos option to work.

\headerbox{}
{name=bottomlogos, span=2, column=0, boxheaderheight=1.5mm, boxColorOne=white, headerColorOne=white, headerColorTwo=white, above=bottom}{
\newlength{\logoheight}
\setlength{\logoheight}{8mm}
\begin{center}
\begin{tabular}{cccccccc}
\includegraphics[height=\logoheight]{Logos/GeneralitatDeCatalunya.png} & 
\includegraphics[height=\logoheight]{Logos/UPC.png} & 
\includegraphics[height=\logoheight]{Logos/Fundacio-CELLEX.png} & 
\includegraphics[height=\logoheight]{Logos/Fundacio-Mir-Puig.png} & 
\includegraphics[height=\logoheight]{Logos/Severo-Ochoa.png} & 
\includegraphics[height=\logoheight]{Logos/MINECO.png} & 
\includegraphics[height=\logoheight]{Logos/UE-FEDER.png} & 
\includegraphics[height=\logoheight]{Logos/ICREA.png} 
\end{tabular}

\vspace{1mm}
\footerfont
ICFO · The Institute of Photonic Sciences | Av. Carl Friedrich Gauss, 3 · Castelldefels · Barcelona
\end{center}
}


%%%%%%%%%%%%%%%%%%%%%%%%%%%%%%%%%%%%%%%%%%%%%%%%%%%%%%%%%%%%%%%%%%%%%%%%%%%%%%%
%%%%   Poster Content  %%%%%%%%%%%%%%%%%%%%%%%%%%%%%%%%%%%%%%%%%%%%%%%%%%%%%%%%
%%%%%%%%%%%%%%%%%%%%%%%%%%%%%%%%%%%%%%%%%%%%%%%%%%%%%%%%%%%%%%%%%%%%%%%%%%%%%%%



\headerbox{%
INTRODUCTION OR ABSTRACT
}{
name=abstract,span=2,column=0,below=titlebox, headerColorOne=white, headerColorTwo=white, boxColorOne=white
}{
Lorem ipsum dolor sit amet, consectetur adipiscing elit. Etiam venenatis convallis laoreet. Proin semper augue faucibus arcu sollicitudin laoreet. In et mi viverra nulla suscipit euismod. Quisque at erat leo. Nullam mollis tortor in arcu suscipit nec venenatis tellus gravida. Nunc in justo quam, et lacinia lacus. Sed rhoncus, odio eu condimentum mattis, nunc massa gravida neque, sollicitudin auctor metus odio in erat. Suspendisse vitae velit eros, vel vehicula quam. Phasellus lorem purus, pellentesque eget pharetra eget, pulvinar a tellus.

\vspace{2mm}
}













\headerbox{%
BACKGROUND
}{name=background,column=0,row=0,below=abstract, }{
\setlength{\parskip}{0.5em}

\vspace{0.1mm}
{\subheaderfont
INULLA ID TURPIS ID NULLA RUTRUM GRAVIDA.
}

Aliquam in massa scelerisque est mollis tempus. Ut non lacus tincidunt sem tincidunt vestibulum. Duis ultricies venenatis ornare. Nam dictum consequat sapien, in facilisis libero tincidunt a. Donec ultricies ornare sapien in tempor.

And, in addition to the boring stuff that PowerPoint can do, \LaTeX{} can do mathematics:
\begin{equation}
V(\mathbf{x}_A,\mathbf{x}_B)=V(\vec{x}_A,\vec{x}_B)=d^2\frac{r^2-2\lambda^2}{(r^2+\lambda^2)^{5/2}},
\end{equation}
where $d$, $\lambda$ and $r$ are symbols, $r=\sqrt{(x_A-x_B)^2+(y_A-y_B)^2}$ has nontrivial inline math, as does the sinning $\sin^2(k_x x)$, but $\chi = h \hbar$ can have some inconsistent formatting in $\hbar$. There's multiple math alphabets,
\begin{equation}
\mathcal{L}(\mathbb{R}, A, \mathsf{A},\alpha|\omega^\Omega, \Gamma, \mathbf{k}, \mathrm{c\cdot D}, \mathfrak{su}\times\mathfrak{B},\mathcal{F}_\mathrm{eff},\hat {T}/\hat{V}, \mathscr{F}, \mathbf{\beta}),
\end{equation}
though \texttt{\textbackslash{}mathscr} requires some attention and  \texttt{\textbackslash{}boldsymbol} can cause some stock warnings.

%\vspace{5mm}
}


\headerbox{}{name=background2, column=0, boxheaderheight=0mm, boxColorOne=icfogray2
, below=background
%, row=0.59 %% use explicit numeric row instead of `below` command to join with the box above. In that case, use an extra \vspace{} on the background box to extend the lower lip.
}{
{\subheaderfont
PROIN AC CONVALLIS SAPIEN.
}

Nullam in urna lacinia orci consequat dictum. Nam scelerisque, nisl in euismod eleifend, justo diam bibendum dui.

\vspace{3mm}
\hspace{1.5mm}
\begin{minipage}{0.95\textwidth}
\begin{multicols}{2}
\includegraphics[width=0.95\columnwidth]{sample-graph.png}
\vfill\null
\columnbreak
\footnotesize
Nullam in urna lacinia orci consequat dictum. Nam scelerisque, nisl in euismod eleifend, justo diam bibendum dui, sed congue diam enim sit amet turpis. Suspendisse rhoncus felis nec lacus blandit mollis. Pellentesque lectus est, accumsan ac pellentesque aliquam, bibendum quis metus. Nunc interdum laoreet odio, eu feugiat velit feugiat sodales. Sed suscipit neque id dui dapibus non facilisis elit ultrices.
\vfill\null
\end{multicols}
\end{minipage}
}


\headerbox{RESULTS}
{name=results, span=1, column=1, row=0, below=abstract, boxColorOne=icfogray3, headerColorOne=black, headerColorTwo=black, headerFontColor=icfogreen}{
\setlength{\parskip}{0.5em}

\begin{multicols}{2}

\small
\color{white}

Bus qui denis incimaio con reserum quamet aliquam re verspernam alias itatur? Qui sinullo ribuste empore ium ipitem rero miligente solupta temodi assequistium quidenis sene quide pore dolupta por modictur aga acienducid moluptate cum vellupta.

\begin{equation*}
\cos\mathopen{}\left(\frac{\omega t}{\sqrt{\alpha\gamma}}\right)\mathclose{} = \sum_{n=0}^\infty \frac{(-1)^{2n}}{(2n)!} \frac{(\omega t)^{2n}}{\alpha^{n}\gamma^{n}}
\end{equation*}



Ectatem pedicimpori re ellenducium et as ex eum fugia prateni con conest qui dolessi andae. Sitaspis earum rem ut dolor alignisci rerovid qui tem faccus nus, ut ab illiber ovitae. Es ad qui blat iliquis et estem fugia dolum sed quias enis dolorerrovid quiscia sus parumqu aecepra voluptur sum doluptatus molorporate vene volupta quia dolorec totatum hil illique ipsum debisimaio eatur maximi. 

\vfill\null
\columnbreak

Volupta nonsequiam vollit volenec eptiisqui odi tem faccus sequi ipsus excerum volut harions. Es ad qui blat iliquis et estem fugia dolum sed quias enis dolorerrovid quiscia.


\begin{center}
\includegraphics[scale=1,valign=T]{sample-figure.pdf}
\end{center}

\footnotesize \smaller
\textbf{Caption:} Graphics produced on Mathematica using MaTeX to make labels and other mathematics match the style of the maths in the poster.
\vfill\null
\end{multicols}

}

\headerbox{REFERENCES}
{name=references,span=1,column=1,below=results}{
Es ad qui blat iliquis et estem fugia dolum sed quias enis dolore rovid quiscia sus parumqu aecepra voluptur sum doluptatus molorporate vene volupta quia dolorec totatum hil illique ipsum debisimaio eatur maximi, volupta nonsequiam vollit volenec eptiisqui odi tem faccus sequi ipsus excerum volut harions.

\begin{enumerate}
\item Qui denis incimaio con reserum quamet
\item Qui sinullo ribuste mporeium ium ipitem rero miligente
\item Solupta temodi assequistium quidenis sequide pore
\item Dolupta por modictur acienducid moluptate cum
\item Vellupta es ad qui blat iliquis et estem fugia
\end{enumerate}
\vspace{1mm}
}





\headerbox{CONCLUSIONS}
{name=conclusions, span=1, column=0, row=0, below=background2, above=bottomlogos}{
\vspace{0.5mm}

\begin{itemize}
\item Aliquam posuere consectetur mi, ac imperdiet lectus auctor non.
Sed commodo posuere nisi, eu rutrum mauris gravida a.
\item Nunc imperdiet sagittis ante vitae elementum.
\item Mauris vel erat vitae nibh tincidunt auctor. In adipiscing vehicula
elementum. Quisque vel est nulla.
\end{itemize}
}

\headerbox{ACKNOWLEDGEMENTS}
{name=acknowledgements, span=1, column=1, below=references, above=bottomlogos}{
\acknowledgementsfont
Qui denis incimaio con reserum quamet. Qui sinullo ribuste mporeium ium ipitem rero miligente. Solupta temodi assequistium quidenis sequide pore Dolupta por modictur acienducid moluptate cum Vellupta es ad qui blat iliquis et.
}

\end{poster}
\end{document}









